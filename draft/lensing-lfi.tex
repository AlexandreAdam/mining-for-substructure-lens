\documentclass[twocolumn]{aastex62}

\usepackage{amsmath, amsthm, amssymb, amsfonts}

\newcommand{\SMS}[1]{{\bf \color{blue}{[SM: #1]}}}

\newcommand{\acronym}[1]{{\small{#1}}}
\newcommand{\package}[1]{\textsl{#1}}
\newcommand{\Euclid}{\textsl{Euclid}}
\newcommand{\lcdm}{\acronym{$\Lambda$CDM}}
\newcommand{\kpc}{\textrm{kpc}}
\newcommand{\Msun}{\textrm{M}_\odot}
\newcommand{\kmps}{\textrm{km\,s}^{-1}}
\newcommand{\zl}{z_l}
\newcommand{\zs}{z_s}
\newcommand{\sv}{\sigma_v}
\newcommand{\mtwo}{m_{200}}
\newcommand{\MMW}{M_\textrm{MW}}
\newcommand{\Mtwo}{M_{200}}
\newcommand{\ctwo}{c_{200}}
\newcommand{\rtwo}{r_{200}}
\newcommand{\Rtwo}{R_{200}}

\shorttitle{Inferring dark matter substructure with likelihood-free inference}
\shortauthors{Brehmer and Mishra-Sharma et al.}
%@arxiver{}

\begin{document}\sloppy\sloppypar\raggedbottom\frenchspacing

\title{\textbf{%
Mining for Substructure: \\
Inferring subhalo population properties from strong lenses with machine learning
}}

% \correspondingauthor{Siddharth Mishra-Sharma}
% \email{sm8383@nyu.edu}

\author{Johann Brehmer}
\affil{Center for Cosmology and Particle Physics, Department of Physics, New York University, 726~Broadway, New York, NY 10003, USA}
\affil{Center for Data Science, New York University, 60 Fifth Ave, New York, NY 10011, USA}

\author{Siddharth Mishra-Sharma}
\affil{Center for Cosmology and Particle Physics, Department of Physics, New York University, 726~Broadway, New York, NY 10003, USA}
\email{sm8383@nyu.edu}

\author{Joeri Hermans}
\affil{University of Li\`ege, Belgium}

\author{Gilles Louppe}
\affil{University of Li\`ege, Belgium}

\author{Kyle Cranmer}
\affil{Center for Cosmology and Particle Physics, Department of Physics, New York University, 726~Broadway, New York, NY 10003, USA}
\affil{Center for Data Science, New York University, 60 Fifth Ave, New York, NY 10011, USA}

\begin{abstract}\noindent
We develop methods to analyze a statistical sample of strong lenses in a principled way to look for dark matter substructure with likelihood-free inference techniques.
\end{abstract}

\keywords{%
cosmology
  ---
dark matter 
  --- 
galaxies: statistics
  ---
gravitational lensing: strong
  ---
methods: statistical 
}

\section{Introduction}
\label{sec:intro}

Dark matter (DM) accounts for nearly a quarter of the energy budget of the Universe, and pinning down its fundamental nature and interactions is one of the most pressing problems in cosmology and particle physics today. Despite an organized effort to do so through terrestrial~\citep{2018PhRvL.121k1302A,2017PhRvL.119r1302C,2017PhRvL.118b1303A}, astrophysical~\citep{2017ApJ...834..110A,2018PhRvD..98l3004C,2018PhRvL.120j1101L} and collider searches~\citep{2019arXiv190301400A,2017PhLB..769..520S}, no conclusive evidence of interactions between the Standard Model (SM) and dark matter exists to-date.

An alternative and complementary approach involves studying dark matter directly through its irreducible gravitational interactions. The concordance Cold Dark Matter (CDM) framework of non-relativistic, collisionless dark matter particles provides an excellent description of the observed distribution of matter on large scales. However, many well-motivated models predict deviations from CDM on smaller scales. Fundamental dark matter microphysical properties, such as its particle mass and self-interaction cross-section, can imprint themselves onto its macroscopic distribution in ways that can be probed by current and future experiments~\citep{2019arXiv190201055D}. As motivating examples, theories where dark matter has a significant free-streaming length would lead to a dearth of subhalos at lower masses ($\lesssim 10^9\,\mathrm{M}_\odot$)~\citep{1983ApJ...274..443B,2001ApJ...556...93B,astro-ph/0004381,0807.0622,1008.0992}, and self-interactions~\citep{1508.03339,1311.6524,1211.6426,1208.3026,1201.5892,1805.03203,1904.10539} or dissipative dynamics~\citep{1706.04195,1702.05482,1707.03829,1303.1521,1512.05349} in the dark sector would modify the structure of the inner core of subhalos as compared to CDM predictions. The existence of ultra-light scalar dark matter~\citep{astro-ph/0003365,2017PhRvD..95d3541H} could also result in interesting phenomenology on small scales where observations do not exist, while being fully consistent with the CDM prediction on larger scales~\citep{1705.05845,1608.02575,2019ApJ...871...28B,1705.05845}.

There exist several avenues for probing the structure of dark matter on small scales. While the detection of ultrafaint dwarf galaxies through the study of stellar overdensities and kinematics~\citep{1503.02584,0706.2687,1503.02079} can be used to make statements about the underlying dark matter properties, theoretical uncertainties in the connection between stellar and halo masses~\citep{1804.03097} and the effect of baryons on the satellite galaxy population~\citep{1812.00044,1811.11791,1701.03792,1608.01849} pose a challenge. Furthermore, suppressed star-formation in smaller halos means that there exists a threshold ($\lesssim 10^8\,\mathrm{M}_\odot$) below which subhalos are expected to be mostly dark and devoid of stars~\citep{1992MNRAS.256P..43E,1611.02281,1607.03127}. This makes studying the imprints of gravitational interaction the \emph{only} viable avenue for probing substructure at smaller scales. In this spirit, the study of perturbations to the stellar phase-space distribution in cold stellar streams~\citep{1804.06854,astro-ph/9807243,1109.6022,1303.4342,1811.03631}, and in stellar fields in the disk and halo~\citep{1711.03554} have been proposed as methods to look for low-mass subhalos through their gravitational interactions in the Milky Way.

Complementary to the study of locally-induced gravitational effects, gravitational lensing has emerged as an important tool for studying the distribution of matter over a large range of scales. Locally, the use of time-domain astrometry has been proposed as a promising method to measure the distribution of local substructure through correlated, lens-induced motions on background celestial objects~\citep{2018JCAP...07..041V}. In the extragalactic regime, galaxy-scale strong lensing systems are a laboratory for studying substructure. The presence of flux-ratio anomalies in multiply-imaged quasar lenses has been used to infer the typical abundance of substructure within galaxy-scale lenses~\citep{2002ApJ...572...25D,1905.04182}. Lensed images of extended sources have been used to find evidence for a handful of subhalos with masses $\gtrsim 10^8\,\mathrm{M}_\odot$~\citep{1601.01388,0910.0760,1201.3643}.

A complementary approach relies on probing the collective effect of sub-threshold (\emph{i.e}, not individually resolvable) subhalos on extended arcs in strongly lensed systems. A particular challenge here is the high dimensionality of the parameter space associated with the large number of subhalos and their (potentially covariant) individual as well as population properties. Methods based on summary statistics~\citep{1702.00009} and studying the amplitude of spatial fluctuations on different scales through power spectra~\citep{1403.2720,1809.00004,1707.04590,1806.07897,1808.03501,1710.03075,1506.01724} have been proposed as ways to reduce the dimensionality of the problem and enable substructure inference in a tractable way. Trans-dimensional techniques may also be able to efficiently map out the parameter space associated with multiple sub-threshold subhalos in these systems~\citep{1508.00662,1706.06111}. This class of methods is well-suited to studying dark matter substructure since they can be sensitive to the \emph{population} properties of low-mass subhalos in strongly lensed galaxies which are directly correlated with the underlying dark matter particle physics. Furthermore, near-future observatories like LSST~\citep{0912.0201,2019arXiv190201055D,1902.05141} and \Euclid~\citep{1001.0061} are expected to find tens of thousands of galaxy-galaxy strong lenses~\citep{2015ApJ...811...20C}, making substructure inference in these systems (and high-resolution followups on a subset) one of the key ways to investigate dark matter substructure and stress-test the Cold Dark Matter paradigm in the near future.

In this paper, we\ldots

This paper is organized as follows.

\section{Strong lensing formalism and simulation set-up}
\label{sec:lensing-formalism}

\subsection{Strong lensing formalism}

In strong lensing systems, the background light emission source can in general be a point-like quasar or supernova, or a faint, extended ``blue'' galaxy. The former results in multiple localized images on the lens plane rather than extended arc-like images, providing the ability to probe substructure over a limited region on the lens plane. For this reason, we focus our method towards galaxy-galaxy lenses---systems producing images with extended arcs---since we aim to disentangle the collective, sub-threshold effect of a population of subhalo perturbers over multiple images. Young, blue galaxies are uniquitous in the redshift regime $z\gtrsim1$ and dominate the faint end of the galaxy luminosity function, resulting in a much larger deliverable sample of galaxy-galaxy strong lenses compared to quadruply- and doubly-imaged quasars/supernovae.

The fact that the strong lens population is expected to be dominated by higher-redshift ($z\gtrsim1$) blue source galaxies lensed by intermediate-redshift ($z\sim 0.5$--$1$) elliptical galaxies presents significant challenges for quantifying the lens population obtainable with future observations. Specifically, planned ground-based surveys like LSST and space telescopes like \Euclid~present complementary challenges for delivering images of strong lensing systems suitable for substructure studies. LSST is expected to image in six bands, allowing efficient source selection and distinguishing source and lens emission, but at the cost of lower resolution by virtue of being a ground-based instrument. \Euclid~imaging is expected be much higher in resolution but with a single optical passband (VIS). Near-IR imaging from WFIRST may deliver a high-resolution, multi-wavelength dataset that is more suitable for substructure studies, although the lens and source populations may differ from those probed by optical telescopes. 

In light of these uncertainties, we limit the scope of the present study to developing a class of methods that can be adapted to the specifications of a galaxy-galaxy strong lensing population obtained with the next generation of optical, near-IR and radio telescopes. In particular, we confine ourselves to a setting where the main methodological points can be made without detailed modeling of the detector capabilities and the deliverable lensing dataset, which is outside of the scope of the current paper.

We now describe our models for the background source, lensing galaxy and population parameters of the lens systems used in this study. 

\subsection{Background source}

We model the emission from background source galaxies using a S\'{e}rsic profile, with the surface brightness given by
\begin{equation}
\Sigma(r)=\Sigma_{e} \exp \left\{-b_{n}\left[\left(\frac{r}{r_{e}}\right)^{1 / n}-1\right]\right\},
\end{equation}
where $r_e$ is the effective circular half-light radius, $n$ is the S\'{e}rsic index, and $b_n$ is a factor depending on $n$ that ensures that $r_e$ contains half the total intensity from the source galaxy, given by~\citep{1999A&A...352..447C}
\begin{align}
b_n \approx 2 n &- \frac{1}{3} + \frac{4}{405 n} + \frac{46}{25515 n^2} \nonumber \\ &+ \frac{131}{1148175 n^3} - \frac{2194697}{30690717750 n^4}. \nonumber
\end{align}

We assume $n=1$ for the source galaxies, corresponding to a flattened exponential profile and consistent with expectation for blue-type galaxies at the relevant redshifts. $\Sigma_{e}$ encodes the flux at half-light radius, which can be mapped onto the total flux (or magnitude) associated with a given galaxy. 

The total unlensed magnitude $M$ (in a given band) of a galaxy can be mapped on to $\Sigma_{e}$ as follows. For a detector with zero-point magnitude $M_0$, which specifies the magnitude of a source giving 1 count\,s$^{-1}$ in expectation, by definition the total counts are given by $S_\mathrm{tot}=10^{0.4(M-M_0)}$. Requiring the half-light radius to contain half the expected counts, for $n=1$ we have the relation $\Sigma_{e} \approx 0.526\,t_\mathrm{exp}S_\mathrm{tot} /(2\pi r_e^2)$ in counts\,arcsec$^{-2}$, where $t_\mathrm{exp}$ is the exposure length.

Treatment of the other S\'{e}rsic parameters, in particular the total emission and half-light radius, in the context of population studies is described in Secs.~\ref{sec:observations} and~\ref{sec:populations} below.

\subsection{Lensing host galaxy}

Cosmological $N$-body simulations suggest that the dark matter distribution in structures at galactic scales can be well-described by a universal, spherically symmetric NFW profile. However, strong lensing probes a region of the host galaxy much smaller than the typical virial radii of galaxy-scale dark matter halo, and the mass budget here is dominated by the baryonic bulge component of the galaxy. Taking this into account, the total mass budget of the lensing host galaxy, being early-type, can be well describe by a singular isothermal ellipsoid (SIE) profile, known as the bulge-halo conspiracy since neither the dark matter nor the baryonic components are individually isothermal. The host profile is thus described as
\begin{equation}
\rho(x, y)=\frac{\sigma_{v}^{2}}{2 \pi G\left(x^{2} / q+q y^{2}\right)}
\label{eq:hostprofile}
\end{equation}
where $\sigma_{v}$ is the central 1-D velocity dispersion of the lens galaxy and $q$ is the ellipsoid axis ratio, with $q=1$ corresponding to a spherical profile. The Einstein radius for this profile, giving the characteristic lensing scale, is given by
\begin{equation}
\theta_{{E}}=4 \pi\left(\frac{\sigma_{v}}{c}\right)^{2} \frac{D_{l s}\left(z_{l}, z_{s}\right)}{D_{s}\left(z_{s}\right)}
\label{eq:siethetae}
\end{equation}
where $D_{ls}$ and $D_s$ are respectively the angular diameter distances from the source to the lens planes and from the source plane to the observer respectively. 

The deflection field for the SIE profile is given by~\citep{2001astro.ph..2341K}
\begin{align} 
\phi_{x} &=\frac{\theta_E q}{\sqrt{1-q^{2}}} \tan ^{-1}\left[\frac{\sqrt{1-q^{2}} x}{\psi}\right] \\ 
\phi_{y} &=\frac{\theta_E q}{\sqrt{1-q^{2}}} \tanh ^{-1}\left[\frac{\sqrt{1-q^{2}} y}{\psi+q^{2} }\right] 
\end{align}
with $\psi \equiv \sqrt{x^2 q^2 + y^2}$.

Although the total galaxy mass (baryons + dark matter) describe the macro lensing field, for the purposes of describing substructure we require being able to map the measure properties of an SIE lens onto the properties of the host dark matter halo. To do this, we relate the central stellar velocity dispersion $\sigma_v$ to the mass $M_{200}$ of the host dark matter halo. \citet{2018ApJ...859...96Z} derived a tight correlation between $\sigma_v$ and $M_{200}$, modeled as
\begin{equation}
\log\left(\frac{M_{200}}{10^{12}\,\Msun}\right) = \alpha + \beta\left(\frac{\sigma_v}{100\,\kmps}\right)
\end{equation}
with $\alpha = 0.09$ and $\beta = 3.48$. % with a mean log-normal scatter of 0.13\,dex. 
We model the host dark matter halo with a Navarro-Frenk-White (NFW) profile~\citep{1996ApJ...462..563N,1997ApJ...490..493N}

\begin{equation}
\rho(r)=\frac{\rho_{s}}{\left(r / r_{s}\right)\left(1+r / r_{s}\right)^{2}}
\label{eq:rhoNFW}
\end{equation}
where $\rho_s$ and $r_s$ are the scale density and scale radius, respectively. The halo virial mass $M_{200}$ describes the total mass contained with the virial radius $r_{200}$, defined as the radius within which the mean density is 200 times the critical density of the universe and related to the scale radius through the concentration parameter $c_{200} \equiv r_{200}/r_s$. Thus, an NFW halo is completely described by the parameters $\{M_{200}, c_{200}\}$. We use the concentration-mass relation from~\citet{2014MNRAS.442.2271S} assuming a log-normal distribution for $c_{200}$ around the median inferred value given by the relation with scatter 0.15\,dex.

The spherically-symmetric deflection for an NFW perturber is given by~\citep{2001astro.ph..2341K}
\begin{equation}
\phi_{r}=4 \kappa_{s} r_{s} \frac{\ln (x / 2)+\mathcal{F}(x)}{x}
\end{equation}
where $x = r/r_s, \kappa_s = \rho_s\,r_s/\Sigma_\mathrm{cr}$ with $\Sigma_\mathrm{cr}$ the critical surface density, and
\begin{equation}
\mathcal{F}(x)=\left\{\begin{array}{ll}{\frac{1}{\sqrt{x^{2}-1}} \tan ^{-1} \sqrt{x^{2}-1}} & {(x>1)} \\ {\frac{1}{\sqrt{1-x^{2}}} \tanh ^{-1} \sqrt{1-x^{2}}} & {(x<1)} \\ {1} & {(x=1).}\end{array}\right.
\label{eq:Fx}
\end{equation}

We described the population parameters we use to model the host velocity dispersion (and thus its Einstein radius and dark matter halo mass) in Secs.~\ref{sec:observations} and~\ref{sec:populations} below.

\subsection{Lensing substructure}

The ultimate goal of our method is to characterize the substructure population in strong lenses. Here we describe our procedure to model the substructure contribution to the lensing signal. Understaing the expected abundance of substructure in galaxies over a large range of epochs is complex undertaking and an active area of research. Properties of individual subhalos (such as their density profiles) as well as those that describe their population (such as the mass and spatial distribution) are strongly affected by their host environment, and accurately modeling all aspects of subhalo evolution and environment is beyond the scope of this paper. Instead, we use simple physically justifiable assumptions to model the substructure contributions in order to highlight the broad methodological points associated with the application of our method.

 \lcdm, often called the standard model of cosmology, predicts a scale-invariant power spectrum of primordial fluctuations and the existence of substrucure over a broad range of masses with equal contribution per logarithmic mass interval. We parameterize the distribution of subhalo masses $\mtwo$ in a given host halo of mass $\Mtwo$---the subhalo mass function---as power law distribution:
\begin{equation}
\frac{M_{200,0}}{\Mtwo}\frac{dn}{d\mtwo} = \alpha\left(\frac{\mtwo}{m_{200, 0}}\right)^{\beta}
\label{eq:shmf}
\end{equation}
where $\alpha$ encodes the overall substructure abundance, with larger $\alpha$ corresponding to more substructure, and the slope $\beta$ encodes the relative constribution of subhalos at different masses, with more negative $\beta$ corresponding to a steeper slope with more low-mass subhalos. The normalization factors $m_{200, 0}$ and $M_{200, 0}$ are arbitrarily set to $10^9\,\Msun$ and the Milky Way mass $\MMW \simeq 1.1\times10^{12}\,\Msun$, respectively.

Theory and simulations within the framework of \lcdm~predict a slope $\beta\sim-0.9$, giving a nearly scale-invariant spectrum of subhalos, which we assume in our fiducial setup. 

% We follow the specifications in~\citet{2016JCAP...09..047H} in order to set the overall fiducial abundance of subhalos, normalizing $\alpha$ to give 150 subhalos in expectation between $10^{8}\,\Msun$ and $10^{10}\,\Msun$ for a Milky Way-sized host halo. 
We parameterize the overall subhalo abundance through the mass fraction contained in subhalos, $f_\mathrm{sub}$, defined as the fraction of the total dark matter halo mass contained in bound substructure in a given mass range. We have
\begin{equation}
f_\mathrm{sub} = \frac{\int_{m_\mathrm{200, min}}^{m_\mathrm{200, max}}d\mtwo\,\mtwo\,\frac{dn}{d\mtwo}}{M_\mathrm{200,host}}
\end{equation}
For a given $f_\mathrm{sub}$, $\beta$ and host halo mass $M_\mathrm{200,host}$, this can be used to determine $\alpha$ in Eq.~\ref{eq:shmf}. The linear scaling of the subhalo mass function with the host halo mass $\Mtwo$ in Eq.~\ref{eq:shmf} is additionally described in~\citet{2016MNRAS.457.1208H,2017MNRAS.469.1997D}. In our fiducial setups, we take the minimum mass $m_\mathrm{200, min} = 10^7\,\Msun$ and $m_\mathrm{200, min} = 0.01\,\,M_\mathrm{200,host}$~\citep{2017MNRAS.469.1997D,2018PhRvD..97l3002H}, and corresponding fiducial substructure fraction in this range of 5\%, roughly consistent with~\citet{2018PhRvD..97l3002H,2019arXiv190504182H,2002ApJ...572...25D} within our considered mass range.

With all parameters of the subhalo mass function specified, the total number $n_\mathrm{tot}$ of subhalos expected within the virial radius $\Rtwo$ of the host halo can be inferred as $\int_{m_\mathrm{200, min}}^{m_\mathrm{200, max}}d\mtwo\,\frac{dn}{d\mtwo}$. Strong lensing probes a region much smaller this scale---the typical Einstein radii for the host deflector are much smaller than the virial radius of the host dark matter halos. In order to obtain the expected number of subhalos within the lensing observations region of interest, we scale the total number of subhalos obtained from the above procedure by the ratio of projected mass within our region of interest $\theta_\textrm{ROI}$ and the host halo mass $\Mtwo$ as follows. We assume the subhalos to be distributed in number density following the host NFW dark matter profile. In this case, the NFW enclosed mass function is $M_\mathrm{enc}(x) = \Mtwo(\ln(x/2) + \mathcal{F}(x))$~\citep{2001astro.ph..2341K}, where $x$ is the angular radius in units of the virial radius, $x\equiv \theta/\theta_{200}$ and $\mathcal{F}$ is given by Eq.~\ref{eq:Fx} above. The number of subhalos within our ROI is thus obtained as $n_\mathrm{ROI} = n_\mathrm{tot} (\ln(x_\mathrm{ROI}/2) + \mathcal{F}(x_\mathrm{ROI}))$. We conservatively take the lensing ROI to enclose a region of angular size twice the Einstein radius of the host halo, $\theta_\mathrm{ROI} = 2\theta_E$.

Since strong lensing probes the line-of-sight distribution of subhalos within the host, their projected spatial distribution is approximately uniform within the lensing ROI~\citep{2017MNRAS.469.1997D}. We thus distribute subhalos uniformly within our ROI. The density profile of subhalos is assumed to be NFW and given by Eq.~\ref{eq:rhoNFW}, with associated lensing properties as described and the concentration inferred from the relation modeled in~\citet{2014MNRAS.442.2271S}.

We finally emphasize that we do not intent to capture all of the intricacies of the subhalo distribution, such as the effects of baryonic physics, tidal distruption of subhalos in proximity to the center of the host and redshift evolution of host as well as substructure properties. Although our description can be extended to take these into account, their precise characterization and effect is still subject to large uncertainties, and our simple model above captures the essential physics for demonstration purposes.

\subsection{Observational considerations}
\label{sec:observations}

A noted above, our method is best-suited to analyzing a statistical sample of strong lenses to search for substructure, such as those that are expected to be obtained in the near future with optical telescopes like \Euclid~and LSST. Given the challenges associated with the precise characterization of such a sample at the present time, we describe here the observational characteristics we assume in order to build up training and testing samples to validate our inference techniques.

We largely follow the description in~\citet{2015ApJ...811...20C}, and use the associated \package{LensPop} package, to characterize our mock observations. In particular, we use the detector configuration for \Euclid, assuming a zero-point magnitude $m_\mathrm{AB} = 25.5$ in the single optical VIS passband, pixel size 0.1\,arcsec, a Gaussian point spread function (PSF) with FWHM 0.18\,arcsec, individual exposures with exposure time 1610\,s, an isotropic sky background with magnitude 22.8\,arcsec$^{-2}$ in the detector passband.

These properties, in particular the exposure, sky background and PSF shape are expected to vary somewhat across the lens sample. Additionally, a given region may be imaged by multiple exposures over a range of color bands. Although these variations can easily be incorporated into our analysis, modeling this is beyond the scope of this study. We briefly comment on how this information can be taken into account later.

\subsection{Population statistics of the lens and source samples}
\label{sec:populations}

A detailed modeling of the lens population discoverable by a future survey like \Euclid~is beyond the scope of the current paper. We rely on~\citet{2015ApJ...811...20C} to describe the population properties of host lens galaxies. In particular, we assume spherical lenses, with ellipticity parameter $q=1$ in Eq.~\ref{eq:hostprofile}. We draw the central 1-D velocity dispersions $\sigma_v$ of host galaxies from a normal distribution with mean 225\,km\,s$^{-1}$ and standard deviation 50\,km\,s$^{-1}$. The results of~\citet{2018ApJ...859...96Z} are used to map the drawn $\sigma_v$ to a dark matter halo mass $\Mtwo$, and the host Einstein radius is analytically inferred with Eq.~\ref{eq:siethetae}.


We draw the lens redshifts $z_l$ from a log-normal distribution with mean 0.56 and scatter 0.25 dex, discarding lenses with $z_l > 1$ as these tend to have a small angular size over which substructure perturbations are relevant. The source offsets $\theta_x$ and $\theta_y$ are drawn from a normal distribution with zero mean and standard deviation 0.2. These are consistent with the lens sample generated from the \package{LensPop} code packaged with~\citet{2015ApJ...811...20C}.

\section{Statistical formalism and likelihood-free inference}
\label{sec:lfi-formalism}

\section{Simulating a population of lenses}
\label{sec:lens-sim}


\section{Results}
\label{sec:results}

\section{Extensions}
\label{sec:extensions}

\section{Conclusions}
\label{sec:conclusions}

\acknowledgements


\software{
\package{Astropy} \citep{2013A&A...558A..33A,2018AJ....156..123A},
\package{IPython} \citep{PER-GRA:2007},
\package{LensPop} \citep{2015ApJ...811...20C},
\package{matplotlib} \citep{Hunter:2007},
\package{NumPy} \citep{numpy:2011},
\package{PyTorch} \citep{paszke2017automatic},
\package{SciPy} \citep{Jones:2001ab}.
}

\bibliographystyle{aasjournal}
\bibliography{lensing-lfi}

\end{document}
